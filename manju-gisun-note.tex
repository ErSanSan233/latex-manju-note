\documentclass{article}
\usepackage{xeCJK}
\usepackage{fontspec}
\usepackage{graphicx}
\usepackage{amssymb}

\usepackage{indentfirst}
\usepackage{atbegshi}
\usepackage{parskip}

\usepackage[normalem]{ulem}  % Provides underline and strikeout commands
\usepackage{tikz}            % Required for drawing the arrow

%输出内容旋转90度
\AtBeginShipout{%
    \global\setbox\AtBeginShipoutBox\vbox{%
        \special{pdf: put @thispage <</Rotate 90>>}%
        \box\AtBeginShipoutBox
    }%
}%

\setCJKmainfont[
    RawFeature={vertical:+vert}, 
    Scale=0.6]
    {NotoSerifCJKtc-Regular}

\setfontfamily\manju[
    Mapping = abkai-to-manju,
    Scale = 1.3]
    {AbkaiXanyan}

\setlength{\parskip}{0.5cm}

%箭头绘制
\newcommand{\arrowuline}[1]{%
  \tikz[baseline=(char.base)]{
    \node[inner sep=0pt,outer sep=0pt] (char) {#1};
    \draw[blue,latex-] ([yshift=0.5ex]char.north west) -- ([yshift=0.5ex]char.north east) |- (char.south east);
  }%
}

\newcommand{\aux}[1]{%
  \!\arrowuline{#1}\!%
}

%双语展示
\newcommand{\bil}[2]{%
  \raisebox{1em}[0pt][0pt]{\makebox[0cm][l]{\footnotesize #2}}\mbox{#1}%
}

%语法成分
\newcommand{\V}{%
  \textcircled{\footnotesize V}\!%
}

\begin{document}
\manju \large 

阅读方向:从右向左(暂时不知道咋设置成从左向右)。

\section{字母}

字母前后接续内容的表格如下:

\begin{tabular}{c|cccccc}
     & a & e & i & o & u & v \\
\hline
k   & $\circ$ & & & $\circ$ & & $\circ$ \\
k''= & & $\circ$ & $\circ$ & & $\circ$ \\
k'=  & $\circ$ & & & $\circ$ & & k'v\\
\hline
=k   & $\circ$ & tek &$\circ$&$\circ$&$\circ$\\
=k'' & k'ak & $\circ$ & & & kuk & $\circ$ \\
\hline
f    & $\circ$ & $\circ$ \\
w    & W & W & F & F & F & F\\
\hline
t    & $\circ$ & & $\circ$ & $\circ$\\
t''  & & $\circ$ & & & $\circ$ & $\circ$
\end{tabular}
\pagebreak

\section{变格}

\begin{des}
    \item[\V =rakv ] 表否定。
    \item[\V =me] 并列副动词。
    \item[\V =ro / \V =ra / \V =re]  \lat{-rA}形,具体由第一和谐律选用。作用:
        \begin{enumerate}
            \item “随意”语气;
            \item 表示将要发生;
            \item 在现在将来时肯定句中与 \bil{uthai}{就} 搭配(对应地,与 \bil{urunakv}{一定} 搭配的句尾应为 =mbi 形);
            \item 作为形动词、动名词充当定语。
        \end{enumerate}
    \item[\V =ha / \V =he / \V =ho / \V =ka / \V =ke / \V =ko] \lat{-HA}形,表过去。由第二和谐律选用。
    \item[\V =nambi / \V =nembi / \V =nombi] 去做\V  ,依第二和谐律选用。
    \item[\V \bil{='jimbi}{\lat{-njimbi}}] 来做\V 。
    \item[\V =qambi / \V =qembi / \V =qombi] 一起做,依第二和谐律选用。
    \item[\V =ndumbi / \V =numbi] 互相做\V ,二者等价。
    \item[\V =qi] 如果做\V。
    \item[\V =kini] 表放任,爱做\V 就让去做去吧。 
\end{des}

\subsection{\lat{-rA}形第一和谐律}

根据\V 结尾音节元音:

\begin{itemize}
    \item o 接 =ro;
    \item a 接 =ra;
    \item 其他接 =re。
\end{itemize}

总结为下表:

\begin{center}
    \begin{tabular}{c|c}
        \toprule
        末音节元音 & \lat{-rA形}\\
        \midrule
        a & \V =ra \\\hline
        e & \multirow{3}{*}{\V =re} \\\cline{1-1}
        i &  \\\hline
        o & \V =ro \\\hline
        u & \multirow{3}{*}{\V =re} \\\cline{1-1}
        v &  \\
        \bottomrule
    \end{tabular}
\end{center}
        
\subsection{\lat{-HA}形第二和谐律} 规则较为复杂。

\begin{enumerate}
    \item \V 为单音节且元音为a / o,接=ha。否则:
    \item \V 末音节元音为o,接=ho;
    \item \V 末音节元音为a,接=ha;
    \item \V 末音节元音为i / v,且倒数第二音节元音为a / i / o / u / v,接=ha;
    \item \V 末音节元音为u,且倒数第二音节元音为a / i / o / v,接=ha;
    \item \V 末音节为二合元音,且前一元音为a / i / o / v,接=ha。除非:
    \item \V 上附加表示使被动的 =bu= 时,仍沿用未添加 =bu= 时的后缀,仅有\V =ho 需变为\V =buha。
    \item 其余接=he。
    \item 特殊除外。
    % \begin{itemize}
    %     \item dosimbi - dosika
    %     \item wesimbi - wesike
    %     \item ombi - oho
    % \end{itemize}
\end{enumerate}

总结为下表:

\begin{center}
    \begin{tabular}{c|c|c|c|c|c|c|c|c}
    \toprule
    \multirow{4}{*}{末音节元音} & \multicolumn{8}{c}{\lat{-HA形}} \\
    % \mrow[3]{末音节元音} & \multicolumn{8}{c}\lat{-HA形} \\
    \cline{2-9} 
    & \multirow{3}{*}{作为双合元音的前者} &  \multicolumn{7}{c}{搭配前音节元音}                        \\ 
    \cline{3-9} 
                  & & 无  & a  & e & i  & o         & u   & v    \\\midrule
    a             &   \multicolumn{8}{c}{\V =ha}                            \\\hline
    e             &  \multicolumn{8}{c}{\V =he}      \\\hline
    i             &  \multirow{3}{*}{\V =ha}       &  \V =he  & \V =ha & \V =he  & \multicolumn{4}{c}{\V =ha}              \\\cline{1-1} \cline{3-9}
    o             &         & \V =ha & \multicolumn{6}{c}{\V =ho (使被动为\V =buha)}                   \\\hline
    u             &  \V =he       & \multirow{3}{*}{\V =he}  & \multirow{3}{*}{\V =ha} & \multirow{3}{*}{\V =he}  & \multicolumn{2}{c|}{\V =ha} &  \V =he   & \V =ha   \\\cline{1-2} \cline{6-9}
    v             &  \V =ha       &   &  &   & \multicolumn{4}{c}{\V =ha}   \\\bottomrule
    \end{tabular}
\end{center}

\subsection{特殊变格的单词总结}

\begin{itemize}
    \item \bil{jembi}{吃} - jeterakv
    \item \bil{ombi}{可以} - ojorakv
    \item \bil{bimbi}{有 / 存在} - bisirakv
    \item \bil{sambi}{知道} - sarkv
\end{itemize}

\subsection{特别小节:两大和谐律中的元音A选用表}

\begin{center}
    \begin{tabular}{c|c|c|c|c|c|c|c|c|c}
        \toprule
        \multirow{4}{*}{末音节元音} & \multirow{4}{*}{Ⅰ} & \multicolumn{8}{c}{Ⅱ} \\
        \cline{3-10} 
        & & \multirow{3}{*}{双合} & \multicolumn{7}{c}{搭配前音节元音} \\ 
        \cline{4-10} 
        & & & 无 & a & e & i & o & u & v \\\midrule
        a & \cellcolor{cyan} a & \cellcolor{cyan} a & \cellcolor{cyan} a & \cellcolor{cyan} a & \cellcolor{cyan} a & \cellcolor{cyan} a & \cellcolor{cyan} a & \cellcolor{cyan} a & \cellcolor{cyan} a \\\hline
        e & \cellcolor{lime} e & \cellcolor{lime} e & \cellcolor{lime} e & \cellcolor{lime} e & \cellcolor{lime} e & \cellcolor{lime} e & \cellcolor{lime} e & \cellcolor{lime} e & \cellcolor{lime} e \\\hline
        i & \cellcolor{lime} e & \cellcolor{cyan} a & \cellcolor{lime} e & \cellcolor{cyan} a & \cellcolor{lime} e & \cellcolor{cyan} a & \cellcolor{cyan} a & \cellcolor{cyan} a & \cellcolor{cyan} a \\\hline
        o* & \cellcolor{pink} o & \cellcolor{cyan} a & \cellcolor{cyan} a & \cellcolor{pink} o & \cellcolor{pink} o & \cellcolor{pink} o & \cellcolor{pink} o & \cellcolor{pink} o & \cellcolor{pink} o \\\hline
        u & \cellcolor{lime} e & \cellcolor{lime} e & \cellcolor{lime} e & \cellcolor{cyan} a & \cellcolor{lime} e & \cellcolor{cyan} a & \cellcolor{cyan} a & \cellcolor{lime} e & \cellcolor{cyan} a \\\hline
        v & \cellcolor{lime} e & \cellcolor{cyan} a & \cellcolor{lime} e & \cellcolor{cyan} a & \cellcolor{lime} e & \cellcolor{cyan} a & \cellcolor{cyan} a & \cellcolor{cyan} a & \cellcolor{cyan} a \\\bottomrule
    \end{tabular}
    
    *注意:使被动形式时需要格外关注o的变化。
\end{center}

% \begin{center}
%     \begin{tabular}{c|c|c|c|c|c|c|c|c|c}
%         \toprule
%         \multirow{4}{*}{末音节元音} & \multirow{4}{*}{第一和谐律} & \multicolumn{8}{c}{第二和谐律} \\
%         \cline{3-10} 
%         & & \multirow{3}{*}{作为双合元音的前者} & \multicolumn{7}{c}{搭配前音节元音} \\ 
%         \cline{4-10} 
%           & & & 无 & a & e & i & o & u & v \\\midrule
%         a & a & a & a & a & a & a & a & a & a \\\hline
%         e & e & e & e & e & e & e & e & e & e \\\hline
%         i & e & a & e & a & e & a & a & a & a \\\hline
%         o & o & a & a & o & o & o & o & o & o \\\hline
%         u & e & e & e & a & e & a & a & e & a \\\hline
%         v & e & a & e & a & e & a & a & a & a \\\bottomrule
%     \end{tabular}
% \end{center}


\pagebreak

\section{句式}

\begin{des}
    \item[元音结尾句提问] 以=v结尾接=n,其余接\bil{=o}{\lat{-u}}。
    \item[\A (oqi) \B inu.] 是。
    \item[\A (oqi) \B waka.] 不是。
    \item[…… bi.] 存在 / 有:batu taqikv de bi.
    \item[…… akv.] 不存在 / 没有:batu taqikv de akv. 此外akv可用于否定形容词:\\
        batu \bil{ahvn}{哥哥} \aux{qi} \bil{den}{高} akv.
    \item[\A \B -i/ni emgi/sasa.] \A 与\B 一起。
    \item[\lat{adj./v.} bime \lat{adj./v.}]  形容词和动词间的“并且”,两侧可以混用。
    \item[\A sere \B ] 称为\A 的\B 。
    \item[\V\ftn{命} nakv] 一做\V 却(强调动作间的转折)
    \item[\A \B be \V\ftn{结句} ayou (sembi).] 表担心, sembi在后面还有单词的时候不能省略。
    \item[(句) na / ne / no / ya ?!] ……吧?! 

    \item[\A \B be 数量 \V=mbi.]数量词放到宾语成分后、动词前。 
    \item[\A -i/ni gubqi] 全\A ,全部\A 。(特殊:全国一般使用 ulusu gurun) 

    \item[\V=rakv ohobi.] 已经不做\V 了。
    \item[动 / 名 / 形 =shvn / =shun] 差一点……
    \item[bahafi \V=mbi] 好不容易做\V / 才得以做\V :\\
        agei amba \bil{algin}{名望} be donjifi \bil{goidaha}{久}, enenggi jabxan de wesihun \bil{qira}{容貌} be bahafi \bil{aqaha}{见面} \bil{de}{表原因} \bil{urgunjehe}{欢喜} seme \bil{wajirakv}{不尽}.
    \item[\V=\ii{=ha}{=he}{=ho} seme \V=rakv] \V 之不\V :\\
        \bil{aliyaha}{后悔} seme \bil{amqarakv}{来得及} 悔之不及 
    \item[\V\ftn{形} turgunde] 因做\V 之故

    \item[…… seqina! / se!(少见)] 那可真得说……! 
    \item[\A be dahame, ……] 跟随\A / 既然\A ,……
    \item[名 / \V\ftn{形} -i/ni teile] 仅……,只……:\\
        \irg bi - bisirei teile 倾其所有 / \bil{mutembi}{能够} - muterei teile 尽其所能  
    \item[ai \V\ftn{形} babi.]有何可\V 之处?
    \item[\A (-i/ni) henduhengge …… sehe.] \A 说过……(用于引述)
    \item[\A hendume …… sembi.] \A 说……(用于描述性地表示)密集对话时,末尾的sembi可省略:
    \begin{itemize}
        \item batu hendume bi generakv sembi.
        \item batu (eme de) fonjime yamji buda ai jembi ? sembi.
    \end{itemize}
    \item[名 -i/ni / \V\ftn{形} + gese / adali]像……一样
    \item[名 -i/ni / \V\ftn{形} + iqi] 沿着,顺着
\end{des}




\subsection{特殊例句辨析}

\paragraph{gurun gvwa sini gebu~be hvlaqi, si je seme \bil{jabuqi}{回答} aqambi.} 其中,gurun取极少使用的“人”的含义。seme 中 se= 充当引号的作用,=me 将其变为副动词。

\paragraph{先后顺序的细微区别}

\begin{itemize}
    \item batu iliha manggi tuqike.
    \item batu ili manggi tuqike.
    \item batu ilime saka / jaka tuqike.
    \item batu ili nakv tuheke.
\end{itemize}

\paragraph{定语从句} 定语从句的主语用属格:

bi (ini minde juwen buhe) bithe be inde be derebumbi.

其中,i minde \bil{juwen}{借} \bil{bu=he}{给-了} 做 bithe 的定语。

batu -i hvlaha bithe be, bi \bil{inu}{也} hvlaha bihe.

\paragraph{“的”字结构动名词}

\begin{itemize}
    \item sikse sini bou de jihe ningge \bil{weqi}{是谁}? 等价于\\
          sikse sini bou de jihengge weqi?
    \item bou -i dorgi de etuku obome bisire ningge, batu inu. 等价于\\
          bou -i dolo etuku obome bisirengge, batu inu. 其中,由于 bi 以 =i 结尾,不能加 ningge ,因此变为 =r=A 形 bisire 再参与变化。
    \item suweni qimari taqikv de genere ningge umesi sain. 等价于\\
          suweni qimari taqikv de generengge umesi sain. 其中主语从句的主语用属格。
\end{itemize}

参考不使用“的”字结构动名词的情况:

sikse taqikv de genehe \underline{niyalma} weqi?

此处 niyalma 就等价于 ninggge 。



\end{document}
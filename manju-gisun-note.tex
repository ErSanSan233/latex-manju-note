\documentclass{article}
% \documentclass{scrartcl}
\usepackage{xeCJK}
\usepackage{fontspec}
\usepackage{graphicx}
\usepackage{amssymb}

\usepackage{indentfirst}
\usepackage{atbegshi}
\usepackage{parskip}

\usepackage[normalem]{ulem}  % Provides underline and strikeout commands
\usepackage{tikz}            % Required for drawing the arrow

\usepackage{multirow}
\usepackage{booktabs}

\usepackage{footnote}
\usepackage{makecell} %表格内换行

%#region 竖排支持
%堆叠行,实现从下向上折行
\XeTeXupwardsmode1

%输出内容旋转90度
\AtBeginShipout{%
    \global\setbox\AtBeginShipoutBox\vbox{%
        \special{pdf: put @thispage <</Rotate 90>>}%
        \box\AtBeginShipoutBox
    }%
}%

%纠正汉字偏移
%作者:Nyoeghau
%链接:https://www.zhihu.com/question/20544732/answer/581301432
\newcommand*\CJKmovesymbol[1]{\raise.45em\hbox{#1}}
\newcommand*\CJKmove{\punctstyle{plain}
    \let\CJKsymbol\CJKmovesymbol
    \let\CJKpunctsymbol\CJKsymbol} %修正baseline
    % \AtBeginDocument{\CJKmove \sloppy}
    \AtBeginDocument{\CJKmove}

%#endregion

%#region 汉满混排字体
\setCJKmainfont[
    RawFeature={vertical:+vert}, 
    Scale=0.7,
    FakeStretch=0.95]
    {NotoSerifCJKsc-Regular}

\setmainfont[
    Mapping = abkai-to-manju,
    Scale=1,
    FakeStretch=1.1]
    {AbkaiXanyan}

%照抄西文
\setfontfamily\latin[
    Scale = 0.7]
    {NotoSerifCJKtc-Regular}

% \setlength{\parskip}{0.5cm}
%#endregion

%#region 自定义命令
%特殊字符的落到CJK字体中
\xeCJKsetcharclass{"24B6}{"24F9}{1}%Ⓐ~ⓩ
% \xeCJKsetcharclass{"002F}{"002F}{1}%/

%箭头绘制
\newcommand{\arrowuline}[1]{%
    \tikz[baseline=(char.base)]{
        \node[inner sep=0pt,outer sep=0pt] (char) {#1};
        \draw[blue,latex-] ([yshift=0.5ex]char.north west) -- ([yshift=0.5ex]char.north east) |- (char.south east);
    }%
}

\newcommand{\aux}[1]{%
    \!\arrowuline{#1}\!%
}

%双语展示
\newcommand{\bil}[2]{
    \raisebox{0.8em}[0pt][0pt]{\makebox[0cm][l]{\tiny #2}}\mbox{#1}
}

%照抄西文
\newcommand{\lat}[1]{
    {\latin #1}
}

%表格合并列
% \newcommand*{\mrow}[2]{
%     \multirow{1+#1}{*}{#2}
% }

%语法成分
\newcommand{\A}{ Ⓐ }
\newcommand{\B}{ Ⓑ }
\newcommand{\C}{ Ⓒ }
\newcommand{\V}{ Ⓥ\hspace{-0.5pt}\ignorespaces}
%#endregion

\begin{document}

\section{字母}

字母前后接续内容的表格如下:

\begin{tabular}{c|cccccc}
     & a & e & i & o & u & v \\
\hline
k   & $\circ$ & & & $\circ$ & & $\circ$ \\
k''= & & $\circ$ & $\circ$ & & $\circ$ \\
k'=  & $\circ$ & & & $\circ$ & & k'v\\
\hline
=k   & $\circ$ & tek &$\circ$&$\circ$&$\circ$\\
=k'' & k'ak & $\circ$ & & & kuk & $\circ$ \\
\hline
f    & $\circ$ & $\circ$ \\
w    & W & W & F & F & F & F\\
\hline
t    & $\circ$ & & $\circ$ & $\circ$\\
t''  & & $\circ$ & & & $\circ$ & $\circ$
\end{tabular}
\pagebreak

\section{词}

oqi:提示助词。

de:与位格格助词,在……、向……、对……。

be:宾经格格助词,把……、将……、经过……。

-i/ni:属用格格助词,……的、用……。
\begin{itemize}
    \item aini整体认读:用什么。
\end{itemize}

qi:从比格格助词,从……。

deri:从比格格助词,穿过。

ele:形容词比较级(增强级)。batu jai tondo, we \aux{ele} \bil{den}{高}? batu \aux{ele} den.

- batu aibi=\aux{=qi} ya ba \aux{be} aibi=\aux{=de} genembi?\\
- batu beging \aux{qi} simiyan \aux{be} halbin \aux{de} genembi.

bi dergi ba \aux{-i} \bil{amba}{大} tang gurun \aux{qi} wargi abka \aux{de} \bil{yargiyan}{真的} \bil{nomun}{经} \aux{be} \bil{bai=\!}{求}\bil{=me}{副动词} genembi.

\subsection{人称代词}

人称代词如下表:

\begin{center}
    \begin{tabular}{c|c|c}
        \toprule
        人称 & 单数 & 复数\\
        \midrule
        \multirow{3}{*}{第一} & \multirow{3}{*}{bi} & be(严格排除对方)\\
            &   &   muse(严格包含对方)\\
        \hline
        第二 & si & suwe\\
        \hline
        第三(正式、尊敬、仅指人) & i & qe\\
        \hline
        第三(随意、不敬、可指物) & tere & tese\\
        \bottomrule
    \end{tabular}
\end{center}

变格时,b= 会变成 n= ,且根据读音规律搭接鼻辅音,即 =be 前加 =m= 、=de / =i / =qi 前接 \bil{='=}{\lat{n}},双音节词除外。全表格如下,后续课文中有分写情况则单独纳入:

\begin{center}
    \begin{tabular}{c|c|c|c|c}
        \toprule
        \multirow{3}{*}{人称代词} & \multicolumn{4}{c}{格和格助词} \\
        \cline{2-5}
            & 属用 -i & 宾经 be & 与位 de & 从比 qi\\
        \midrule
        bi & mini & mimbe & minde & minqi \\
        be & meni & membe & mende & menqi \\
        \hline
        muse & musei & musebe / muse be & musede & museqi \\
        \hline
        si & sini & simbe & sinde & sinqi \\
        suwe & suweni & suwembe & suwende & suwenqi \\
        \hline
        i & ini & imbe & inde & inqi \\
        qe & qeni & qembe & qemde & qenqi \\
        \hline
        tere & terei & terebe & terede & tereqi \\
        tese & tesei & tesebe & tesede & teseqi \\
        \bottomrule
    \end{tabular}
\end{center}

\subsection{特别小节:不规则动词整理}

\begin{des}
    \item[bimbi] bisirakv 
    \item[dosimbi] dosika 
    \item[jembi] jefu! - jeterakv - jekenembi / jekenjimbi - jeke / jebuhe
    \item[jimbi] jiu! - jiderakv
    \item[sambi] sarkv 
    \item[ombi] oso! - ojorakv - ojorou - oho
    \item[wesimbi] wesike 
\end{des}
\pagebreak

\section{变格}

\begin{des}
    \item[\V =rakv ] 表否定。
    \item[\V =me] 并列副动词。
    \item[\V =ro / \V =ra / \V =re]  \lat{-rA}形,具体由第一和谐律选用。作用:
        \begin{enumerate}
            \item “随意”语气;
            \item 表示将要发生;
            \item 在现在将来时肯定句中与 \bil{uthai}{就} 搭配(对应地,与 \bil{urunakv}{一定} 搭配的句尾应为 =mbi 形);
            \item 作为形动词、动名词充当定语。
        \end{enumerate}
    \item[\V =ha / \V =he / \V =ho / \V =ka / \V =ke / \V =ko] \lat{-HA}形,表过去。由第二和谐律选用。
    \item[\V =nambi / \V =nembi / \V =nombi] 去做\V  ,依第二和谐律选用。
    \item[\V \bil{='jimbi}{\lat{-njimbi}}] 来做\V 。
    \item[\V =qambi / \V =qembi / \V =qombi] 一起做,依第二和谐律选用。
    \item[\V =ndumbi / \V =numbi] 互相做\V ,二者等价。
    \item[\V =qi] 如果做\V。
    \item[\V =kini] 表放任,爱做\V 就让去做去吧。 
\end{des}

\subsection{\lat{-rA}形第一和谐律}

根据\V 结尾音节元音:

\begin{itemize}
    \item o 接 =ro;
    \item a 接 =ra;
    \item 其他接 =re。
\end{itemize}

总结为下表:

\begin{center}
    \begin{tabular}{c|c}
        \toprule
        末音节元音 & \lat{-rA形}\\
        \midrule
        a & \V =ra \\\hline
        e & \multirow{3}{*}{\V =re} \\\cline{1-1}
        i &  \\\hline
        o & \V =ro \\\hline
        u & \multirow{3}{*}{\V =re} \\\cline{1-1}
        v &  \\
        \bottomrule
    \end{tabular}
\end{center}
        
\subsection{\lat{-HA}形第二和谐律} 规则较为复杂。

\begin{enumerate}
    \item \V 为单音节且元音为a / o,接=ha。否则:
    \item \V 末音节元音为o,接=ho;
    \item \V 末音节元音为a,接=ha;
    \item \V 末音节元音为i / v,且倒数第二音节元音为a / i / o / u / v,接=ha;
    \item \V 末音节元音为u,且倒数第二音节元音为a / i / o / v,接=ha;
    \item \V 末音节为二合元音,且前一元音为a / i / o / v,接=ha。除非:
    \item \V 上附加表示使被动的 =bu= 时,仍沿用未添加 =bu= 时的后缀,仅有\V =ho 需变为\V =buha。
    \item 其余接=he。
    \item 特殊除外。
    % \begin{itemize}
    %     \item dosimbi - dosika
    %     \item wesimbi - wesike
    %     \item ombi - oho
    % \end{itemize}
\end{enumerate}

总结为下表:

\begin{center}
    \begin{tabular}{c|c|c|c|c|c|c|c|c}
    \toprule
    \multirow{4}{*}{末音节元音} & \multicolumn{8}{c}{\lat{-HA形}} \\
    % \mrow[3]{末音节元音} & \multicolumn{8}{c}\lat{-HA形} \\
    \cline{2-9} 
    & \multirow{3}{*}{作为双合元音的前者} &  \multicolumn{7}{c}{搭配前音节元音}                        \\ 
    \cline{3-9} 
                  & & 无  & a  & e & i  & o         & u   & v    \\\midrule
    a             &   \multicolumn{8}{c}{\V =ha}                            \\\hline
    e             &  \multicolumn{8}{c}{\V =he}      \\\hline
    i             &  \multirow{3}{*}{\V =ha}       &  \V =he  & \V =ha & \V =he  & \multicolumn{4}{c}{\V =ha}              \\\cline{1-1} \cline{3-9}
    o             &         & \V =ha & \multicolumn{6}{c}{\V =ho (使被动为\V =buha)}                   \\\hline
    u             &  \V =he       & \multirow{3}{*}{\V =he}  & \multirow{3}{*}{\V =ha} & \multirow{3}{*}{\V =he}  & \multicolumn{2}{c|}{\V =ha} &  \V =he   & \V =ha   \\\cline{1-2} \cline{6-9}
    v             &  \V =ha       &   &  &   & \multicolumn{4}{c}{\V =ha}   \\\bottomrule
    \end{tabular}
\end{center}

\subsection{特殊变格的单词总结}

\begin{itemize}
    \item \bil{jembi}{吃} - jeterakv
    \item \bil{ombi}{可以} - ojorakv
    \item \bil{bimbi}{有 / 存在} - bisirakv
    \item \bil{sambi}{知道} - sarkv
\end{itemize}

\subsection{特别小节:两大和谐律中的元音A选用表}

\begin{center}
    \begin{tabular}{c|c|c|c|c|c|c|c|c|c}
        \toprule
        \multirow{4}{*}{末音节元音} & \multirow{4}{*}{Ⅰ} & \multicolumn{8}{c}{Ⅱ} \\
        \cline{3-10} 
        & & \multirow{3}{*}{双合} & \multicolumn{7}{c}{搭配前音节元音} \\ 
        \cline{4-10} 
        & & & 无 & a & e & i & o & u & v \\\midrule
        a & \cellcolor{cyan} a & \cellcolor{cyan} a & \cellcolor{cyan} a & \cellcolor{cyan} a & \cellcolor{cyan} a & \cellcolor{cyan} a & \cellcolor{cyan} a & \cellcolor{cyan} a & \cellcolor{cyan} a \\\hline
        e & \cellcolor{lime} e & \cellcolor{lime} e & \cellcolor{lime} e & \cellcolor{lime} e & \cellcolor{lime} e & \cellcolor{lime} e & \cellcolor{lime} e & \cellcolor{lime} e & \cellcolor{lime} e \\\hline
        i & \cellcolor{lime} e & \cellcolor{cyan} a & \cellcolor{lime} e & \cellcolor{cyan} a & \cellcolor{lime} e & \cellcolor{cyan} a & \cellcolor{cyan} a & \cellcolor{cyan} a & \cellcolor{cyan} a \\\hline
        o* & \cellcolor{pink} o & \cellcolor{cyan} a & \cellcolor{cyan} a & \cellcolor{pink} o & \cellcolor{pink} o & \cellcolor{pink} o & \cellcolor{pink} o & \cellcolor{pink} o & \cellcolor{pink} o \\\hline
        u & \cellcolor{lime} e & \cellcolor{lime} e & \cellcolor{lime} e & \cellcolor{cyan} a & \cellcolor{lime} e & \cellcolor{cyan} a & \cellcolor{cyan} a & \cellcolor{lime} e & \cellcolor{cyan} a \\\hline
        v & \cellcolor{lime} e & \cellcolor{cyan} a & \cellcolor{lime} e & \cellcolor{cyan} a & \cellcolor{lime} e & \cellcolor{cyan} a & \cellcolor{cyan} a & \cellcolor{cyan} a & \cellcolor{cyan} a \\\bottomrule
    \end{tabular}
    
    *注意:使被动形式时需要格外关注o的变化。
\end{center}

% \begin{center}
%     \begin{tabular}{c|c|c|c|c|c|c|c|c|c}
%         \toprule
%         \multirow{4}{*}{末音节元音} & \multirow{4}{*}{第一和谐律} & \multicolumn{8}{c}{第二和谐律} \\
%         \cline{3-10} 
%         & & \multirow{3}{*}{作为双合元音的前者} & \multicolumn{7}{c}{搭配前音节元音} \\ 
%         \cline{4-10} 
%           & & & 无 & a & e & i & o & u & v \\\midrule
%         a & a & a & a & a & a & a & a & a & a \\\hline
%         e & e & e & e & e & e & e & e & e & e \\\hline
%         i & e & a & e & a & e & a & a & a & a \\\hline
%         o & o & a & a & o & o & o & o & o & o \\\hline
%         u & e & e & e & a & e & a & a & e & a \\\hline
%         v & e & a & e & a & e & a & a & a & a \\\bottomrule
%     \end{tabular}
% \end{center}


\section{句式}

\begin{itemize}
    \item A B -i/ni emgi/sasa :A与B一起。
\end{itemize}

\end{document}
\documentclass{article}
\usepackage{xeCJK}
\usepackage[a5paper]{geometry}
\usepackage{graphicx}

\usepackage{indentfirst}
\usepackage{atbegshi}

\usepackage{fancyhdr}


\usepackage[normalem]{ulem}  % Provides underline and strikeout commands
\usepackage{tikz}            % Required for drawing the arrow

\usepackage{multirow}
\usepackage{booktabs}
\usepackage{makecell} %表格内换行

\usepackage{xcolor}
\usepackage{colortbl}

%#region 竖排支持
%堆叠行,实现从下向上折行

\geometry{left=2.5cm, right=4cm, top=2cm, bottom=3cm}

\XeTeXupwardsmode1

%输出内容旋转90度
\AtBeginShipout{%
    \global\setbox\AtBeginShipoutBox\vbox{%
        \special{pdf: put @thispage <</Rotate 90>>}%
        \box\AtBeginShipoutBox
    }%
}%

%纠正汉字偏移
%作者:Nyoeghau
%链接:https://www.zhihu.com/question/20544732/answer/581301432
\newcommand*\CJKmovesymbol[1]{\raise.45em\hbox{#1}}
\newcommand*\CJKmove{\punctstyle{plain}
    \let\CJKsymbol\CJKmovesymbol
    \let\CJKpunctsymbol\CJKsymbol} %修正baseline
    % \AtBeginDocument{\CJKmove \sloppy}
    \AtBeginDocument{\CJKmove}

%#endregion

%#region 汉满混排字体
\setCJKmainfont[
    RawFeature={vertical:+vert}, 
    Scale=0.7,
    FakeStretch=0.95]
    {NotoSerifCJKsc-Regular}

\setmainfont[
    Mapping = abkai-to-manju,
    % Scale=1,
    FakeStretch=1.13,
    BoldFont=AbkaiXanyan]
    {AbkaiBulekuBithe}

\renewcommand{\thesection}{\textcolor{blue}{\arabic{section}}}
\renewcommand{\thesubsection}{\textcolor{blue}{\arabic{section}-\arabic{subsection}}}
\renewcommand{\thesubsubsection}{\textcolor{blue}{\arabic{section}-\arabic{subsection}-\arabic{subsubsection}}}

%照抄西文
\setfontfamily\latin[
    Scale = 0.7]
    {NotoSerifCJKtc-Regular}

\setlength{\parindent}{0pt}
\setlength{\parskip}{0.5em}
%#endregion

%#region 自定义命令

%特殊字符的落到CJK字体中
\xeCJKsetcharclass{"24B6}{"24F9}{1}%Ⓐ~ⓩ
% \xeCJKsetcharclass{"002F}{"002F}{1}%/

%箭头绘制
\newcommand{\arrowuline}[1]{%
    \tikz[baseline=(char.base)]{
        \node[inner sep=0pt,outer sep=0pt] (char) {#1};
        \draw[blue,latex-] ([yshift=0.5ex]char.north west) -- ([yshift=0.5ex]char.north east) |- (char.south east);
    }%
}

\newcommand{\aux}[1]{%
    \!\arrowuline{#1}\!%
}

%特殊符号
\newcommand{\irg}{%
    \textcolor{red}{※}}

%双语展示
\newcommand{\bil}[2]{%
    \raisebox{0.8em}[0pt][0pt]{\makebox[0cm][l]{\tiny #2}}\mbox{#1}%
}

%照抄西文
\newcommand{\lat}[1]{
    {\latin #1}
}

\newcommand{\ftn}[1]{
    {\footnotesize #1}
}

%语法成分
\newcommand{\A}{ Ⓐ }
\newcommand{\B}{ Ⓑ }
\newcommand{\C}{ Ⓒ }
\newcommand{\V}{ Ⓥ\hspace{-0.5pt}\ignorespaces}

\newcommand{\red}[1]{%
    \textcolor{red}{#1}%
}

\newcommand{\blue}[1]{%
    \textcolor{blue}{#1}%
}

\newcommand{\ii}[3]{%
    \raisebox{1.6\baselineskip}{\makebox[0cm]{\hspace{1mm}\tiny Ⅱ}}%
    \fbox{\hspace{-0.35em}%
    \raisebox{-0.7\baselineskip}{\makebox[0cm][l]{#1}}%
    \raisebox{0.7\baselineskip}{\makebox[0cm][l]{#3}}%
    #2%
}}

\newcommand{\HA}{%
    \fbox{HA}%
}

% \newcommand{\HAfina}{%
%     \fbox{HA}
%     % =\makebox{\raisebox{1.6\baselineskip}{\makebox[0cm]{\hspace{0.8em}\tiny Ⅱ}}%
%     % \fbox{\makebox[2.2em]{%
%     % \raisebox{-0.7\baselineskip}{\makebox[0cm][l]{=ha / =ka}}%
%     % \raisebox{0.7\baselineskip}{\makebox[0cm][l]{=ho / =ko}}%
%     % \raisebox{0.25\baselineskip}{{=he / =ke}}%
%     % }}}
% }

% \newcommand{\HAmedi}{%
%     =\makebox{\raisebox{1.2\baselineskip}{\makebox[0cm]{\hspace{0.8em}\tiny (\!\raisebox{0.3mm}{\lat{o}}\!)Ⅱ}}%
%     \fbox{\makebox[0.2em]{%
%     \makebox[0cm]{==he==}%
%     \raisebox{-0.5\baselineskip}{\makebox[0cm]{=ha==}}%
%     \raisebox{0.5\baselineskip}{\makebox[0cm]{=ho=}}%
%     }}
% }\!}

\newcommand{\AIfina}{%
    \raisebox{1.2\baselineskip}{\makebox[0cm]{\hspace{1mm}\tiny Ⅰ}}%
    \fbox{\makebox[0.1em]{%
    \makebox[0cm]{=e }%
    \raisebox{-0.6\baselineskip}[1ex][1.5mm]{\makebox[0cm]{=a }}%
    \raisebox{0.6\baselineskip}[3.8mm][1ex]{\makebox[0cm]{=o }}%
    }}
}

\newcommand{\AImedi}{%
    \raisebox{1\baselineskip}{\makebox[0cm]{\hspace{1mm}\tiny Ⅰ}}%
    \fbox{\makebox[-0.1em]{%
    \raisebox{-0.1\baselineskip}{\makebox[0cm]{=e==}}%
    \raisebox{-0.6\baselineskip}[1ex][1.5mm]{\makebox[0cm]{=a==}}%
    \raisebox{0.5\baselineskip}[3mm][1ex]{\makebox[0cm]{=o==}}%
    }}}

\newcommand{\AIIfina}{%
    \raisebox{1.2\baselineskip}{\makebox[0cm]{\hspace{1mm}\tiny Ⅱ}}%
    \fbox{\makebox[0.1em]{%
    \makebox[0cm]{=e }%
    \raisebox{-0.6\baselineskip}[1ex][1.5mm]{\makebox[0cm]{=a }}%
    \raisebox{0.6\baselineskip}[3.8mm][1ex]{\makebox[0cm]{=o }}%
    }}
}

\newcommand{\AIImedi}{%
    \raisebox{1\baselineskip}{\makebox[0cm]{\hspace{1mm}\tiny Ⅱ}}%
    \fbox{\makebox[-0.1em]{%
    \raisebox{-0.1\baselineskip}{\makebox[0cm]{=e==}}%
    \raisebox{-0.6\baselineskip}[1ex][1.5mm]{\makebox[0cm]{=a==}}%
    \raisebox{0.5\baselineskip}[3mm][1ex]{\makebox[0cm]{=o==}}%
    }}}

%列表成分
\newenvironment{des}{
    \begin{list}{}{%
        \renewcommand{\makelabel}[1]{% 生成项目标签:
                    【{\textbf{##1}}】}}}
{\end{list}}
%#endregion

\begin{document}

\fancyhf{}
\pagestyle{fancy}
\lhead{\thepage}%左侧
\chead{}%中间
\rhead{}
\lfoot{}
\cfoot{}%当前页
\rfoot{}

\renewcommand{\headrulewidth}{0mm}
\renewcommand{\footrulewidth}{0mm}

{\large
\noindent jalan~-i siden~de mini xanggiyan alin bi,\\
alin~-i hanqi sahaliyan ula eyer tugi~-i adali.\\
alin muke~de musei mafari banjihabi,\\
mukvn~-i enen amba kesi~be alimbi

arxan burge~de fulehe akvqi, tere uthai gargan banjirakv. ere uthai gaxan~be kidure akaqun inu, ere uthai gaxan~be kidure gvnin inu.

\noindent jalan~-i siden de mini xanggiyan alin bi,\\
alin~-i hanqi sahaliyan ula eyer tugi~-i adali.\\
muse ainu mafari gaxan~be waliyambi,\\
enenggi mafari gaxan~be hargaxambi.

kituhan~de berge akvqi, tere uthai uqulerakv. ere uthai gaxan be kidule akaqun inu, ere uthai gaxan~be kidure gvnin inu.}

\pagebreak

\section{字母}

字母前后接续内容的表格如下:

\begin{tabular}{c|cccccc}
     & a & e & i & o & u & v \\
\hline
k   & $\circ$ & & & $\circ$ & & $\circ$ \\
k''= & & $\circ$ & $\circ$ & & $\circ$ \\
k'=  & $\circ$ & & & $\circ$ & & k'v\\
\hline
=k   & $\circ$ & tek &$\circ$&$\circ$&$\circ$\\
=k'' & k'ak & $\circ$ & & & kuk & $\circ$ \\
\hline
f    & $\circ$ & $\circ$ \\
w    & W & W & F & F & F & F\\
\hline
t    & $\circ$ & & $\circ$ & $\circ$\\
t''  & & $\circ$ & & & $\circ$ & $\circ$
\end{tabular}
\pagebreak

\section{词}

oqi:提示助词。

de:与位格格助词,在……、向……、对……。

be:宾经格格助词,把……、将……、经过……。

-i/ni:属用格格助词,……的、用……。
\begin{itemize}
    \item aini整体认读:用什么。
\end{itemize}

qi:从比格格助词,从……。

deri:从比格格助词,穿过。

ele:形容词比较级(增强级)。batu jai tondo, we \aux{ele} \bil{den}{高}? batu \aux{ele} den.

- batu aibi=\aux{=qi} ya ba \aux{be} aibi=\aux{=de} genembi?\\
- batu beging \aux{qi} simiyan \aux{be} halbin \aux{de} genembi.

bi dergi ba \aux{-i} \bil{amba}{大} tang gurun \aux{qi} wargi abka \aux{de} \bil{yargiyan}{真的} \bil{nomun}{经} \aux{be} \bil{bai=\!}{求}\bil{=me}{副动词} genembi.

\subsection{人称代词}

人称代词如下表:

\begin{center}
    \begin{tabular}{c|c|c}
        \toprule
        人称 & 单数 & 复数\\
        \midrule
        \multirow{3}{*}{第一} & \multirow{3}{*}{bi} & be(严格排除对方)\\
            &   &   muse(严格包含对方)\\
        \hline
        第二 & si & suwe\\
        \hline
        第三(正式、尊敬、仅指人) & i & qe\\
        \hline
        第三(随意、不敬、可指物) & tere & tese\\
        \bottomrule
    \end{tabular}
\end{center}

变格时,b= 会变成 n= ,且根据读音规律搭接鼻辅音,即 =be 前加 =m= 、=de / =i / =qi 前接 \bil{='=}{\lat{n}},双音节词除外。全表格如下,后续课文中有分写情况则单独纳入:

\begin{center}
    \begin{tabular}{c|c|c|c|c}
        \toprule
        \multirow{3}{*}{人称代词} & \multicolumn{4}{c}{格和格助词} \\
        \cline{2-5}
            & 属用 -i & 宾经 be & 与位 de & 从比 qi\\
        \midrule
        bi & mini & mimbe & minde & minqi \\
        be & meni & membe & mende & menqi \\
        \hline
        muse & musei & musebe / muse be & musede & museqi \\
        \hline
        si & sini & simbe & sinde & sinqi \\
        suwe & suweni & suwembe & suwende & suwenqi \\
        \hline
        i & ini & imbe & inde & inqi \\
        qe & qeni & qembe & qemde & qenqi \\
        \hline
        tere & terei & terebe & terede & tereqi \\
        tese & tesei & tesebe & tesede & teseqi \\
        \bottomrule
    \end{tabular}
\end{center}

\subsection{特别小节:不规则动词整理}

\begin{des}
    \item[bimbi] bisirakv 
    \item[dosimbi] dosika 
    \item[jembi] jefu! - jeterakv - jekenembi / jekenjimbi - jeke / jebuhe
    \item[jimbi] jiu! - jiderakv
    \item[sambi] sarkv 
    \item[ombi] oso! - ojorakv - ojorou - oho
    \item[wesimbi] wesike 
\end{des}
\pagebreak

\section{动词}

\subsection{动词变化}

四个特殊的动词:inu、waka、bi、akv。可以认为它们无形态变化。

\subsubsection{现在时}

\begin{des}
    \item[\V=mbi] 动词原形,现在将来、客观真理。
    \item[\V=r=\AIfina]  即\lat{-rA}形,具体由第一和谐律选用。作用:
        \begin{enumerate}
            \item “随意”语气;
            \item 表示将要发生;
            \item 作为形动词、动名词充当定语。
        \end{enumerate}
    \item[\V=rakv ] 对现在将来的否定。
    \item[\V=\HA bi] 现在完成,已经做\V  
    \item[配合副动词]见下:
    \begin{des}
        \item[并列:\V=me bi] 现在进行,正在做\V 。否定为\V=me bisirakv。
        \item[持续:\V=\HA=i bi] 现在持续,一直做\V 。强调动作的全过程。
        \item[顺序:\V=fi bi] 仍然在做\V 。强调动作的收尾或动作的影响,不关心动作中间的细节。
    \end{des} 
\end{des} 

\subsubsection{过去}
\begin{des}
    \item[\V=\ii{=ha}{=he}{=ho} / \V=\ii{=ka}{=ke}{=ko}] \lat{-HA}形,一般过去,做了\V 。由第二和谐律选用。否定为\V=\HA akv / \V=\HA=kv。
    \item[\V=\HA bihe (bi)] 过去完成,曾做过\V 。否定为\V=\HA=kv bihe (bi)。
    \item[\V=mbihe] 过去惯常,动词原型的过去时,表示曾惯常的动作。否定为\V=rakv bihe (bi)。
    \begin{itemize}
        \item bi beging de bihe \bil{fon}{时} de, \bil{erdedari}{每早} erde buda jembihe.
    \end{itemize}
    \item[配合副动词]见下:
    \begin{des}
        \item[并列:\V=me bihe] 过去进行,曾在做\V 。否定为\V=me bihekv。
        \item[假设:\V=qi aqambihe] 本应做\V 。
    \end{des} 
\end{des} 

\subsubsection{命令、祈请、意愿}
\begin{des}
    \item[\V !] 命令式。
    \item[\A \V=qina!] \A ,你就做\V 吧!
    \item[\V=ki!]
    \item[\A , \B be \V\ftn{命} se!] (不客气)小\A ,你去让\B 做\V !
    \item[\V=kini] 表放任,爱做\V 就让去做去吧。 
    \item[(\A ,)(\B be) \V=kini!] 祝\B ……。其中\A 代表一种超自然力量:
    \begin{itemize}
        \item \bil{enduri}{神}, batu be \bil{banjiha}{生} inenggi \bil{urgunjekini}{高兴}.
    \end{itemize}
    \item[\A , \B be \V=kini!] (客气)\A ,请你请\B 做\V :
    \begin{itemize}
        \item batu, sini ama be qimari taqikv de jikini.
    \end{itemize}
    \item[\V=r=\AIfina=u!]
    \item[(bi) \V=ki.] 
    \item[\V=ki bai!] 咱们一起做\V 吧!
    \item[\A \V=ki sembi.] \A 想做\V 。
    \item[\A \B be \V=kini sembi.] \A 想让\V 做\V :
    \begin{itemize}
        \item ama batu be bithe hvlanakini sembi.
    \end{itemize}
\end{des} 

否定形式:

\begin{des}
    \item[ume \V=r=\AIfina !] 别做\V 。
    \item[\V=rakv ojorou!] 请勿做\V 。 
\end{des}

\subsubsection{虚拟}

\begin{des}
    \item[\A \B (be) \V=rahv (sembi).] \A 担心\B 做\V 。sembi 位于句尾时可省略,后面还有单词时不能省略。
    \item[\A \B (be) \V=rahv ayou (sembi).] 与上等价。
    \item[\A \B (be) \V=rakv ojorahv sembi.]  \A 恐怕\B 不做\V 。
    \item[\A \B (be) \V=rakv ayou sembi.] 与上等价。 
    \begin{itemize}
        \item batu tondo be qimari generahv.
        \item bati tondo be qimari genere ayou (sembi).
        \item batu tondo be qimari generahv seme ofi, i uthai geneki serakv ohobi.
        \item batu tondo be qimari genere ayou seme ofi, i uthai geneki serakv ohobi.
    \end{itemize}
\end{des}

\subsubsection{态}

\begin{des}
    \item[使被动] 如下。注意直接宾语和间接宾语的配合:\A \B be \C de \bil{bumbi}{给}中,\B 为直接宾语,\C 为间接宾语。
    \begin{des}
        \item[\A \B \bil{be}{让} {[\C (be)]} \V=bumbi.] 使动:\A 让 \B 做 \V \C 。
        \item[\A \B \bil{de}{被} (\C) \V=bumbi.] 被动。 \C 极少出现。
        \item[\A \B \bil{be}{让 / 把} \C \bil{de}{被 / 给} \V=bumbi.] 使被动。
    \end{des}
    \begin{itemize}
        \item ama batu \aux{be} efen jebumbi.
        \item efen bati \aux{de} jebumbi.
        \item ama efen \aux{be} batu \aux{de} jebumbi. 爸爸把点心给巴图吃 / 爸爸使点心被巴图吃。
    \end{itemize}
    \item[\V=n=\AIImedi=mbi] 去做\V ,依第二和谐律选用。
    \item[\V \bil{='jimbi}{\lat{-njimbi}}] 来做\V 。\irg jembi
    \item[\V=q=\AIImedi=mbi] 一起做\V ,依第二和谐律选用。在久远时此与 \V=ndumbi / \V=numbi 混用,亦不区分含义。
    \item[\V=ndumbi / \V=numbi] 互相做\V ,二者等价。
    \item[\V=q==\AIImedi =mbi/ \V=j=\AIImedi =mbi/ \V=x=\AIImedi =mbi/ \V=\ii{=tambi}{=tembi}{=tombi} / \V=\ii{=dambi}{=dembi}{=dombi}] 反复做\V ,经常做\V ,总是做\V ,频繁做\V 。
\end{des}

\subsection{副动词}

副动词不能结句。

\begin{des}
    \item[\V=me] 并列副动词,做着\V 。
    \item[\V=fi] 顺序副动词,做\V 之后。 
    \item[\V=qi] 条件/假设副动词,如果做\V 。
    \item[\V=qibe] 让步副动词,虽然……,即使……:
        \begin{itemize}
            \item udu …… \V=qibe
            \item udu …… seme
        \end{itemize}
    \item[\V=\HA=i] 持续副动词,一直做\V 。
    \item[\V=\ii{=tala}{=tele}{=tolo}] “直到”副动词,直到……才:
        \begin{itemize}
            \item \bil{xun}{太阳} \bil{tuhetele}{落下} ama \bil{teni}{才} \bil{bederehe}{回}.
            \item ama \bil{tantatala}{打} batu teni \bil{kaiqarakv}{喊}. 
        \end{itemize}
    \item[\V=\ii{=tai}{=tei}{=toi}] 极尽副动词,玩儿命做\V :
        \begin{itemize}
            \item ama batu be tantatai, batu \bil{songgohoi}{哭}, eme bederetele, ama teni \bil{nakaha}{停止}.
            \item \bil{simnen}{考试} \bil{hami=\!\red{=ka}}{\irg 临近} \bil{turgunde}{因为}, batu inenggidari \bil{buqetei}{死亡} taqimbi.
        \end{itemize}
    \item[\V=pi / \V=mpi] 延伸副动词,逐渐做\V :
        \begin{itemize}
            \item \bil{meifen}{脖子} \bil{伸}{sampi} tuwambi.
        \end{itemize}
    \item[\V=r=\AIfina onggolo] 未完成副动词,做\V 之前。等价于\V=\ii{=nggala}{=nggele}{=nggolo} (第二和谐律):
        \begin{itemize}
            \item taqinggala uthai bahanambi.
            \item genenggele \bil{afini}{早已} sahabi.
            \item songgonggolo \bil{juqutun}{电视} tuwame bihe.
        \end{itemize}
    \item[\V=r=\AImedi=lame] 伴随副动词,一边……一边……
    \item[\V=shun / \V=shvn / \V=meliyan / \V=liyan] 程度副动词,稍微……:
    \begin{itemize}
        \item batu \bil{injemeliyan}{笑} \bil{gisurembi}{说}.
    \end{itemize} 
    选取=shun / =shvn的规则如下表:
    \begin{center}
        \begin{tabular}{c|c}
            \toprule
            末音节元音 & \lat{伴随副动词后缀}\\
            \midrule
            a & =shvn \\
            e & =shun \\
            i & =shvn \\
            o & =shvn \\
            u & =shun \\
            v & =shvn \\
            \bottomrule
        \end{tabular}
    \end{center}
\end{des}

\subsection{能愿动词}

\begin{des}
    \item[\V=qi ombi] 可以做\V ,指被允许。
    \item[\V=qi aqambi] 应该做\V 。
    \item[\V=me mutembi] 能做\V ,指客观条件。
    \item[\V=me bahanambi] 会做\V ,指学习过。
\end{des}

\subsection{所有型形容动词}

\begin{des}
\item[\V\ftn{形} ele] 等价于\V=l=\AIfina (实际只对应 a / e / o / v 的情况),所有做\V 的。
\begin{itemize}
    \item sikse taqikv de genehe ele taqisi \bil{youni}{全都} \bil{tuqinju}{出来}!
    \item sikse taqikv de genehele taqisi youni tuqinju!
    \item sikse taqikv de genehelengge youni tuqinju!
\end{itemize}
\end{des}

\subsection{“的”字结构动名词}

\begin{des}
    \item[\V\ftn{形} ningge / \V\ftn{形}\!=ngge ] 指“\V 的事”或“\V 的人”。
    \item[\V\ftn{形} ele ningge / \V\ftn{形}=l=\AIfina ningge / \V\ftn{形}=l=\AImedi=ngge] 所有型“的”字结构动名词。
\end{des}

\begin{itemize}
    \item sikse sini bou de jihe ningge \bil{weqi}{是谁}? 等价于\\
          sikse sini bou de jihengge weqi?
    \item bou -i dorgi de etuku obome bisire ningge, batu inu. 等价于\\
          bou -i dolo etuku obome bisirengge, batu inu. 其中,由于 bi 以 =i 结尾,不能加 ningge ,因此变为 =r=A 形 bisire 再参与变化。
    \item suweni qimari taqikv de genere ningge umesi sain. 等价于\\
          suweni qimari taqikv de generengge umesi sain. 其中主语从句的主语用属格。
\end{itemize}

参考不使用“的”字结构动名词的情况:

sikse taqikv de genehe \underline{niyalma} weqi?

此处 niyalma 就等价于 ninggge 。

\subsection{助动词}

此节罗列助动词的变化。

\subsubsection{bimbi}

指有、在。

\begin{des}
    \item[形 / 名 bihe.] 曾是……
        \begin{itemize}
            \item batu \bil{daqi}{原本} taqisi bihe.
            \item tere ilha daqi fulgiuyan bihe.
        \end{itemize}
\end{des}

\subsubsection{ombi}

指成为。

\begin{des}
    \item[\A \B de \C ombi.] \A 对于\B 来说是\C ,\A 是\B 的\C 。
        \begin{itemize}
            \item erimbu batu de \bil{ahvn}{哥} ombi.
            \item si minde \bil{ergen}{生命} -i dorgi \bil{ulden}{晨光} ombi.
        \end{itemize}
    \item[\A \B de ombi] 见下:
    \begin{des}
        \item[\A 与\B 合得来] erimbu batu de ombi, damu tondo de ojorakv.
        \item[\A 成为\B 的] 见下:
            \begin{itemize}
                \item tere bithe batu de ohobi.
                \item qimari si minde ombi kai.
            \end{itemize}
        \item[到(时间 / 地点)时]见下:
            \begin{itemize}
                \item tere \bil{fon}{时} de ojo=\!\red{\bil{=ro kakade}{的情况下}} ainara?
                \item taqikv de oho manggi jai muke omiki!
            \end{itemize}
    \end{des}
    \item[\A \B ombi] \A 成为\B :
        \begin{itemize}
            \item batu jidere aniya taqisi ombi.
            \item tere niyalma taqibusi oho.
        \end{itemize}
    \item[\A \B oho \C] \A 是成了\B 的\C :
        \begin{itemize}
            \item batu oqi, taqibusi oho niyalma.
        \end{itemize}
    \item[\A ojoro \B] 作为\A 的\B :
        \begin{itemize}
            \item taqibusi ojoro niyalma, taqisi be \bil{karmaqi}{保护} aqambi.
            \item taqibusi ojorongge, taqisi be karmaqi aqambi.
        \end{itemize}
    \item[\A \B be \C obumbi] \A 把\B 变成 / 当作\C :
        \begin{itemize}
            \item batu muke be \bil{juhe}{冰} obuha.
            \item batu \bil{holo}{假} gisun be \bil{yargiyan}{真} gisun obume \bil{donjiha}{听}
        \end{itemize}
    \item[配合时间词的变化]见下:
        \begin{des}
            \item[ome] 将到,刚到,刚过
            \item[ofi] 刚过,之后
            \item[ohobi] 已经……了
            \item[ohakv] 没过(表示比预期时间短) 
        \end{des}
        \begin{itemize}
            \item juwe biya ome \bil{nenden}{梅} ilha \bil{ilahabi}{开放}.
            \item ilan inenggi ofi ama \bil{bederehe}{回}.
            \item \bil{herqun akv}{不知不觉间}, \bil{usilan dube}{周末} ohobi.
            \item \bil{udu usilan}{几周} ohakv(或 oho akv,与中文相同,用“没过”表示比预期短,但实际已经过了), terei mafa akv oho.
        \end{itemize}
    \item[\V=rakv / \V=me / akv + ombi] 表示强调,可就……了:
        \begin{itemize}
            \item bi \bil{jai jai}{再也} \red{\bil{ere=}{期望}}\!=rakv ombi
        \end{itemize}
\end{des}

\subsubsection{sembi}

指说 / 叫作。

\begin{des}
    \item[\A be \B sembi.] 把\A 叫 / 称作\B 。
        \begin{itemize}
            \item ere be fi sembi.
            \item gebu be batu sembi.
            \item gebu be tangsu sere niyalma \bil{labdu}{多} bi.
        \end{itemize}
    \item[\A \B be …… sembi.] \A 说\B ……。
        \begin{itemize}
            \item batu ama be bou de akv sembi.
            \item taqibusi batu be \bil{boui urebun}{家庭作业} arahakv sehe.
            \item \bil{hvjasi}{警察} tere niyalma be niyalma \bil{waha}{杀} sembi.
        \end{itemize}
    \item[“说了”类词…… sembi]见下:
        \begin{des}
            \item[hendume] 说(\A hendume …… sembi. \A 说……(用于描述性地表示)密集对话时,末尾的sembi可省略:)
            \begin{itemize}
                \item batu hendume bi generakv sembi.
                \item batu (eme de) fonjime yamji buda ai jembi ? sembi.
            \end{itemize}
            \item[hendume gisun] 说了的话
            \item[gisun henduhengge] 话说了(\A (-i/ni) henduhengge …… sehe. \A 说过……(用于引述) )
            \item[fonjime] 问
            \item[donjiqi] 听说
            \item[dekderi gisun] 谚语说  
        \end{des}
        \begin{itemize}
            \item batu hendume bi \bil{yali}{肉} jeki sembi.
            \item kungzi -i henduhengge ere \bil{umai}{并(用于否定)} hvwanggiyarakv sehebi / sehe bihe.
            \item donjiqi qimari aga agambi sembi.
            \item dekdeni gisun \bil{ulgiyan}{猪} -i \bil{oforo}{鼻} de \bil{elu}{葱} \bil{插}{sisime} \bil{sufan}{象} \bil{arambi}{写 / 做 / 过(节) / 伪装} sembi.
        \end{itemize}
    \item[\A …… seme \V\ftn{主要}] \A 说着……做\V\ftn{主要} :
        \begin{itemize}
            \item ama ainu sain -i taqirakv seme, batu -i \bil{ura}{尻} be \bil{tantambi}{打}.
            \item ama batu de aibide genehe? seme fonjiqi, batu tondo -i bou de \bil{efime}{玩} genehe seme \bil{jabuha}{回答}.
            \item eme batu de \bil{juqutun}{电视} de ume tuwara! seme \bil{afabuha}{嘱咐}.
            \item gurun gvwa sini gebu~be hvlaqi, si je seme \bil{jabuqi}{回答} aqambi. 其中,gurun取极少使用的“人”的含义。seme 中 se= 充当引号的作用,=me 将其变为副动词。
        \end{itemize}
    \item[\A be …… seme ……] 因为\A ……
        \begin{itemize}
            \item batu ama be \bil{se}{年龄} baha seme beging de generakv. (se bahambi 上年纪)
            \item taqibusi batu de boui urebun de arahakv seme batui ama be taqikv~de jubuhe.
        \end{itemize}
    \item[(udu / uthai) …… seme / seqibe (inu) ……] 就算……也,即使……也,虽然……也
        \begin{itemize}
            \item udu \bil{tuweri}{冬} seme batu inu \bil{jiramin}{厚} \bil{etuki}{衣} eturakv.
            \item uthai taqibusi jihe seme batu inu \bil{gelerakv}{怕}.
        \end{itemize}
    \item[摹拟词接sembi各种形式] 表示状态:
        \begin{des}
            \item[…… sembi] \bil{bira}{河} -i muke hvwalar hvwalar sembi.
            \item[…… seme] 做状语,接形容词或动词:bira -i muke hvwalar hvwalar seme \bil{eyembi}{流淌}.
            \item[…… sere] 现在将来形容动词,做谓语,接名词:hvwalar hvwalar sere bira -i muke
            \item[…… serengge] hvwalar hvwalar serengge, bira -i muke inu.
        \end{des}
        \begin{itemize}
            \item abka \bil{qak seme}{极寒状} \bil{beikuwen}{寒冷}.
            \item \bil{dur sere}{蜂拥状} taqisisa \bil{tuqika}{出}.
        \end{itemize}
\end{des}

\subsection{\lat{-rA}形扩展}

在现在将来时肯定句中与 \bil{uthai}{就} 搭配(对应地,与 \bil{urunakv}{一定} 搭配的句尾应为 =mbi 形);

\begin{des}
    \item[\V=r=\AIfina unde] 尚未做。
    \item[\V=r=\AIfina de amuran] 喜欢做\V ,爱好。
    \begin{itemize}
        \item \bil{muji}{大麦} \bil{nure}{非蒸馏酒} omirelame \bil{juqutun}{电视} tuwara de amuran.
    \end{itemize} 
    \item[\V=r=\AIfina de mangga] 做\V 是困难的
    \item[\V=r=\AIfina de ja]做\V 是容易的
    \item[\V=r=\AIfina mangga / \V=r=\AImedi=ngge mangga] 善于做\V  
    \item[\V=r=\AIfina jakade] 因为做\V / 做\V 的时候 / ……情况下 / 条件下。注意:jakade 前必须用\lat{-rA}形,与具体时态无关。
    \item[\V=r=\AIfina dabala] 等价于\V=mbi dere:罢了,不过是。 
\end{des}

\subsection{\lat{-HA}形扩展}

\begin{des}
    \item[\V=\HA manggi] 做\V 之后
\end{des}

\subsection{=me副动词扩展}

\begin{des}
    \item[\V=me saka / jaka] 等价于\V (命) manggi ,一\V 之后马上就……(强调二者紧接着发生)
\end{des}

\subsection{其他}

\begin{des}
    \item[\V=\ii{=ta}{=te}{=to}] 直到。依第二和谐律选用。
    \item[名=\ii{=ngga}{=ngge}{=nggo}] 变形容词:\\
        \bil{algin}{名气} - \bil{algingga}{有名的} / 
        \bil{gebu}{名字} - \bil{gebungge}{有名的} / 
        \bil{boqo}{颜色} - \bil{boqonggo}{彩色的}
    \item[动 / 名 =si] 做……的人:\\
        \bil{muqe}{锅} - \bil{muqesi}{厨师} /
        \bil{sejen}{汽车} - \bil{sejesi}{司机} /
        \bil{buyembi}{爱} - \bil{buyendusi}{爱人} /
        \bil{takambi}{认识} - \bil{takandusi}{熟人}、\bil{takandubusi}{介绍人}
\end{des}

\subsection{\lat{-rA}形第一和谐律}

根据\V 结尾音节元音:

\begin{itemize}
    \item o 接 =ro;
    \item a 接 =ra;
    \item 其他接 =re。
\end{itemize}

总结为下表:

\begin{center}
    \begin{tabular}{c|c}
        \toprule
        末音节元音 & \lat{-rA形}\\
        \midrule
        a & \V=ra \\\hline
        e & \multirow{3}{*}{\V=re} \\\cline{1-1}
        i &  \\\hline
        o & \V=ro \\\hline
        u & \multirow{3}{*}{\V=re} \\\cline{1-1}
        v &  \\
        \bottomrule
    \end{tabular}
\end{center}
        
\subsection{\lat{-HA}形第二和谐律} 规则较为复杂。

\begin{enumerate}
    \item \V 为单音节且元音为a / o,接=ha。否则:
    \item \V 末音节元音为o,接=ho;
    \item \V 末音节元音为a,接=ha;
    \item \V 末音节元音为i / v,且倒数第二音节元音为a / i / o / u / v,接=ha;
    \item \V 末音节元音为u,且倒数第二音节元音为a / i / o / v,接=ha;
    \item \V 末音节为二合元音,且前一元音为a / i / o / v,接=ha。除非:
    \item \V 上附加表示使被动的 =bu= 时,仍沿用未添加 =bu= 时的后缀,仅有\V=ho 需变为\V=buha。
    \item 其余接=he。
    \item 特殊除外。
\end{enumerate}

总结为下表:

\begin{center}
    \begin{tabular}{c|c|c|c|c|c|c|c|c}
    \toprule
    \multirow{4}{*}{末音节元音} & \multicolumn{8}{c}{\lat{-HA形}} \\
    % \mrow[3]{末音节元音} & \multicolumn{8}{c}\lat{-HA形} \\
    \cline{2-9} 
    & \multirow{3}{*}{作为双合元音的前者} &  \multicolumn{7}{c}{搭配前音节元音}                        \\ 
    \cline{3-9} 
                  & & 无  & a  & e & i  & o         & u   & v    \\\midrule
    a             &   \multicolumn{8}{c}{\V=ha}                            \\\hline
    e             &  \multicolumn{8}{c}{\V=he}      \\\hline
    i             &  \multirow{3}{*}{\V=ha}       &  \V=he  & \V=ha & \V=he  & \multicolumn{4}{c}{\V=ha}              \\\cline{1-1} \cline{3-9}
    o             &         & \V=ha & \multicolumn{6}{c}{\V=ho (使被动为\V=buha)}                   \\\hline
    u             &  \V=he       & \multirow{3}{*}{\V=he}  & \multirow{3}{*}{\V=ha} & \multirow{3}{*}{\V=he}  & \multicolumn{2}{c|}{\V=ha} &  \V=he   & \V=ha   \\\cline{1-2} \cline{6-9}
    v             &  \V=ha       &   &  &   & \multicolumn{4}{c}{\V=ha}   \\\bottomrule
    \end{tabular}
\end{center}

\subsection{特别小节:两大和谐律中的元音A选用表}

\begin{center}
    \begin{tabular}{c|c|c|c|c|c|c|c|c|c}
        \toprule
        \multirow{4}{*}{末音节元音} & \multirow{4}{*}{Ⅰ} & \multicolumn{8}{c}{Ⅱ} \\
        \cline{3-10} 
        & & \multirow{3}{*}{双合} & \multicolumn{7}{c}{搭配前音节元音} \\ 
        \cline{4-10} 
        & & & 无 & a & e & i & o & u & v \\\midrule
        a & \cellcolor{cyan} a & \cellcolor{cyan} a & \cellcolor{cyan} a & \cellcolor{cyan} a & \cellcolor{cyan} a & \cellcolor{cyan} a & \cellcolor{cyan} a & \cellcolor{cyan} a & \cellcolor{cyan} a \\\hline
        e & \cellcolor{lime} e & \cellcolor{lime} e & \cellcolor{lime} e & \cellcolor{lime} e & \cellcolor{lime} e & \cellcolor{lime} e & \cellcolor{lime} e & \cellcolor{lime} e & \cellcolor{lime} e \\\hline
        i & \cellcolor{lime} e & \cellcolor{cyan} a & \cellcolor{lime} e & \cellcolor{cyan} a & \cellcolor{lime} e & \cellcolor{cyan} a & \cellcolor{cyan} a & \cellcolor{cyan} a & \cellcolor{cyan} a \\\hline
        o* & \cellcolor{pink} o & \cellcolor{cyan} a & \cellcolor{cyan} a & \cellcolor{pink} o & \cellcolor{pink} o & \cellcolor{pink} o & \cellcolor{pink} o & \cellcolor{pink} o & \cellcolor{pink} o \\\hline
        u & \cellcolor{lime} e & \cellcolor{lime} e & \cellcolor{lime} e & \cellcolor{cyan} a & \cellcolor{lime} e & \cellcolor{cyan} a & \cellcolor{cyan} a & \cellcolor{lime} e & \cellcolor{cyan} a \\\hline
        v & \cellcolor{lime} e & \cellcolor{cyan} a & \cellcolor{lime} e & \cellcolor{cyan} a & \cellcolor{lime} e & \cellcolor{cyan} a & \cellcolor{cyan} a & \cellcolor{cyan} a & \cellcolor{cyan} a \\\bottomrule
    \end{tabular}
    
    *注意:使被动形式时需要格外关注o的变化。
\end{center}

% \begin{center}
%     \begin{tabular}{c|c|c|c|c|c|c|c|c|c}
%         \toprule
%         \multirow{4}{*}{末音节元音} & \multirow{4}{*}{第一和谐律} & \multicolumn{8}{c}{第二和谐律} \\
%         \cline{3-10} 
%         & & \multirow{3}{*}{作为双合元音的前者} & \multicolumn{7}{c}{搭配前音节元音} \\ 
%         \cline{4-10} 
%           & & & 无 & a & e & i & o & u & v \\\midrule
%         a & a & a & a & a & a & a & a & a & a \\\hline
%         e & e & e & e & e & e & e & e & e & e \\\hline
%         i & e & a & e & a & e & a & a & a & a \\\hline
%         o & o & a & a & o & o & o & o & o & o \\\hline
%         u & e & e & e & a & e & a & a & e & a \\\hline
%         v & e & a & e & a & e & a & a & a & a \\\bottomrule
%     \end{tabular}
% \end{center}


\section{句式}

\begin{itemize}
    \item A B -i/ni emgi/sasa :A与B一起。
\end{itemize}

\end{document}
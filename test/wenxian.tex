\documentclass{article}
\usepackage{xeCJK}
\usepackage{fontspec}
\usepackage{atbegshi}
\usepackage{gezhu}
\usepackage{makecell}
\usepackage{geometry}

\XeTeXupwardsmode1

\setCJKmainfont[
    RawFeature={vertical:+vert}, 
    Scale=0.7,
    FakeStretch=0.95]
    {NotoSerifCJKsc-Regular}

\setmainfont[
    Mapping = ../abkai-to-manju,
    Scale = 1]
    {AbkaiXanyan}

\setgezhulines{2} 

% \setgezhuraise{-2pt}



%输出内容旋转90度
\AtBeginShipout{%
    \global\setbox\AtBeginShipoutBox\vbox{%
        \special{pdf: put @thispage <</Rotate 90>>}%
        \box\AtBeginShipoutBox
    }%
}%

\geometry{
    paperheight = 12.25cm,
    paperwidth = 16cm,
    % inner = 2cm,
    % outer = 2cm,
    left = 3cm,
    right = 2cm,
    top = 2cm,
    bottom = 2cm
}

%双语展示
\newcommand{\bilr}[2]{
    \raisebox{0.5em}{\makebox[0cm][l]{\small #2}}\mbox{#1}
}

\newcommand{\bill}[2]{
    \raisebox{-0.4em}{\makebox[0cm][l]{\tiny #2}}\mbox{#1}
}

\newcommand{\slb}[2]{
    \makebox[0cm][l]{\raisebox{0.3em}{#1}}\raisebox{-0.3em}{#2}
}

\newcommand{\ct}[4]{%词头
    \everygezhu{\fontsize{18}{4.8}\selectfont}
    \gezhu{\bill{#3}{#4} \linebreak[4] \bilr{#1}{#2}}
}

\newcommand{\sy}[1]{%释义
    \everygezhu{\fontsize{15}{6.5}\selectfont}
    \gezhu{#1}
}

% \linespread{3}
\renewcommand{\baselinestretch}{6}

\begin{document}
\begin{withgezhu}
    \hspace{-1.5cm}
    \ct{天}{tyan}{abka}{阿補\slb 喀阿} 
    \sy{umesi den tumen jaka be elbehengge be, abka sembi.}
    \ct{上天}{xang tian}{dergi abka}{\slb 德额哷\slb 基伊 阿補\slb 喀阿}
    \sy{tumen jaka be elbehe be jorime gisurembihede, dergi abka sembi.}
    \ct{蒼天}{cang tian}{niuhon abka}{\slb 尼伊優\slb 和鄂安 阿補\slb 喀阿}
    \sy{abkai boqo be jorime  gisurembihede niuhon abka sembi.}
    \ct{清天}{qubg tiyan}{genggiyen abka}{\slb 哥額鞥\slb 基伊\slb 葉額恩 阿補\slb 喀阿}
    \sy{tugi sukdun akv umesi bolgo getugen be, genggiyen abka sembi.}
    \ct{天氣清肅}{tiyan qi qing su}{abka fundehun}{阿補\slb 喀阿 \slb 扶烏恩\slb 德額\slb ?烏恩}
    \sy{boloti ome akbai boqo seksehun ojoro be, abka fundehun sembi.}
    \ct{晨光}{qen guwang}{ulden}{烏勒\slb 德額恩}
    \sy{alin jakarara onggolo tuqire elden be ulden sabumbi sembi.}
    \ct{晨光現出}{qen guwang hiyan qu}{uldeke}{烏勒\slb 德額\slb 珂額}
    \sy{ulden tuqike be uldeke sembi.}
\end{withgezhu}
\end{document}


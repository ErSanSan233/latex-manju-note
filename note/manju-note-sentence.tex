\pagebreak

\section{句式}

\begin{des}
    \item[元音结尾句提问] 以=v结尾接=n,其余接\bil{=o}{\lat{-u}}。
    \item[\A (oqi) \B inu] 是。
    \item[\A (oqi) \B waka] 不是。
    \item[…… bi] 存在 / 有:batu taqikv de bi.
    \item[…… akv] 不存在 / 没有:batu taqikv de akv. 此外akv可用于否定形容词,如batu \bil{ahvn}{哥哥} \aux{qi} \bil{den}{高} akv.
    \item[\A \B -i/ni emgi/sasa.] \A 与\B 一起。
    \item[\lat{adj./v.} bime \lat{adj./v.}]  形容词和动词间的“并且”,两侧可以混用。
    \item[\A sere \B ] 称为\A 的\B 。
    \item[\A \V =qina] \A ,你就做\V 吧!
    \item[\V\ftn{命} nakv] 一做\V 却(强调动作间的转折)
    \item[\A \B (be) \V =rahv (sembi).] \A 担心\B 做\V 。 
    \item[\A \B be \V \ftn{结句} ayou (sembi).] 表担心, sembi在后面还有单词的时候不能省略。
    \item[(句) na / ne / no / ya ?!] ……吧?! 
    \item[\V\ftn{形} ningge / \V\ftn{形}\!=ngge ] “的”字结构动名词,指“\V 的事”或“\V 的人”。
    \item[\A \B be 数量 \V =mbi.]数量词放到宾语成分后、动词前。 
    \item[\A -i/ni gubqi] 全\A ,全部\A 。(特殊:全国一般使用 ulusu gurun) 
    \item[=ki 和 sembi] 见下
    \begin{des}
        \item[\A \V =ki sembi.] \A 想做\V 。
        \item[\V =ki bai!] 咱们一起做\V 吧! 
        \item[\A \B be \V =kini sembi.] \A 想让\V 做\V 。 
        \item[(\A ,)(\B be) \V =kini.] 祝\B ……。其中\A 代表一种超自然力量。
        \item[\A , \B be \V =kini.] (客气)\A ,请你请\B 做\V 。
        \item[\A , \B be \V \ftn{命} se!] (不客气)小\A ,你去让\B 做\V !
    \end{des}
    \item[\V =rakv ohobi.] 已经不做\V 了。
\end{des}

\subsection{=r=A形扩展}

\begin{des}
    \item[\V =r=A unde] 尚未做。
    \item[ume \V =r=A] 别做\V 。
\end{des}

\subsection{=HA形扩展}

\begin{des}
    \item[\V =HA bihe] 曾做过。
    \item[\V =HA akv / \V =HA=kv] 没做过。
    \item[\V =HA bi] 已经做了\V 。现在时完成体? 
    \item[\V =HA manggi] 做\V 之后
    \item[\V =HA=i] 一直做,为持续副动词,不能结句
\end{des}

\subsection{=me扩展}

\begin{des}
    \item[\V =me bi] 正在做。
    \item[\V =me bihe] 曾在做。
    \item[\V =me saka / jaka] 等价于\V (命) manggi ,一\V 之后马上就……(强调二者紧接着发生)
\end{des}

\subsection{使被动}

使动:\A \B \bil{be}{让} [\C (be)] \V =bumbi. \A 让 \B 做 \V \C 。

被动:\A \B \bil{de}{被} (\C) \V =bumbi. \C 极少出现

使被动:\A \B \bil{be}{让 / 把} \C \bil{de}{被 / 给} \V =bumbi. 

注意COD和COI的配合:\A \B be \C de \bil{bumbi}{给}中,\B 为COD,\C 为COI。

\begin{itemize}
    \item ama batu \aux{be} efen jebumbi.
    \item efen bati \aux{de} jebumbi.
    \item ama efen \aux{be} batu \aux{de} jebumbi. 爸爸把点心给巴图吃 / 爸爸使点心被巴图吃。
\end{itemize}

\subsection{情态动词}

\begin{des}
    \item[\V =qi ombi] 可以做\V ,指被允许。
    \item[\V =qi aqambi] 应该做\V 。
    \item[\V =me mutembi] 能做\V ,指客观条件。
    \item[\V =me bahanambi] 会做\V ,指学习过。
\end{des}


\subsection{例句辨析}

\paragraph{gurun gvwa sini gebu~be hvlaqi, si je seme \bil{jabuqi}{回答} aqambi.} 其中,gurun取极少使用的“人”的含义。seme 中 se= 充当引号的作用,=me 将其变为副动词。

\paragraph{先后顺序的细微区别}

\begin{itemize}
    \item batu iliha manggi tuqike.
    \item batu ili manggi tuqike.
    \item batu ilime saka / jaka tuqike.
    \item batu ili nakv tuheke.
\end{itemize}

\paragraph{定语从句} 定语从句的主语用属格:

bi (ini minde juwen buhe) bithe be inde be derebumbi.

其中,i minde \bil{juwen}{借} \bil{bu=he}{给-了} 做 bithe 的定语。

batu -i hvlaha bithe be, bi \bil{inu}{也} hvlaha bihe.

\paragraph{“的”字结构动名词}

\begin{itemize}
    \item sikse sini bou de jihe ningge \bil{weqi}{是谁}? 等价于\\
          sikse sini bou de jihengge weqi?
    \item bou -i dorgi de etuku obome bisire ningge, batu inu. 等价于\\
          bou -i dolo etuku obome bisirengge, batu inu. 其中,由于 bi 以 =i 结尾,不能加 ningge ,因此变为 =rA 形 bisire 再参与变化。
    \item suweni qimari taqikv de genere ninggge umesi sain. 等价于\\
          suweni qimari taqikv de generenggge umesi sain. 其中主语从句的主语用属格。
\end{itemize}

参考不使用“的”字结构动名词的情况:

sikse taqikv de genehe \underline{niyalma} weqi?

此处 niyalma 就等价于 ninggge 。

\paragraph{=rakv表担心} 例句:

\begin{itemize}
    \item batu tondo be qimari generahv.
    \item bati tondo be qimari genere ayou (sembi).
    \item batu tondo be qimari generahv seme ofi, i uthai geneki serakv ohobi.
    \item batu tondo be qimari genere ayou seme ofi, i uthai geneki serakv ohobi.
\end{itemize}



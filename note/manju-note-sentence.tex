\pagebreak

\section{句式}

\begin{des}
    \item[元音结尾句提问] 以=v结尾接=n,其余接\bil{=o}{\lat{-u}}。
    \item[\A (oqi) \B inu.] 是。
    \item[\A (oqi) \B waka.] 不是。
    \item[…… bi.] 存在 / 有:batu taqikv de bi.
    \item[…… akv.] 不存在 / 没有:batu taqikv de akv. 此外akv可用于否定形容词:\\
        batu \bil{ahvn}{哥哥} \aux{qi} \bil{den}{高} akv.
    \item[\A \B -i/ni emgi/sasa.] \A 与\B 一起。
    \item[\lat{adj./v.} bime \lat{adj./v.}]  形容词和动词间的“并且”,两侧可以混用。
    \item[\A sere \B ] 称为\A 的\B 。
    \item[\V\ftn{命} nakv] 一做\V 却(强调动作间的转折)
    \item[(句) na / ne / no / ya ?!] ……吧?! 
    \item[\A \B be 数量 \V=mbi.]数量词放到宾语成分后、动词前。 
    \item[\A -i/ni gubqi] 全\A ,全部\A 。(特殊:全国一般使用 ulusu gurun) 
    \item[\V=rakv ohobi.] 已经不做\V 了。
    \item[动 / 名 / 形 =shvn / =shun] 差一点……
    \item[bahafi \V=mbi] 好不容易做\V / 才得以做\V :
    \begin{itemize}
        \item agei amba \bil{algin}{名望} be donjifi \bil{goidaha}{久}, enenggi jabxan de wesihun \bil{qira}{容貌} be bahafi \bil{aqaha}{见面} \bil{de}{表原因} \bil{urgunjehe}{欢喜} seme \bil{wajirakv}{不尽}.
    \end{itemize}
    \item[\V=\ii{=ha}{=he}{=ho} seme \V=rakv] \V 之不\V :
    \begin{itemize}
        \item \bil{aliyaha}{后悔} seme \bil{amqarakv}{来得及} 悔之不及 
    \end{itemize}
    \item[\V\ftn{形} turgunde] 因做\V 之故
    \item[…… seqina! / se!(少见)] 那可真得说……! 
    \item[\A be dahame, ……] 跟随\A / 既然\A ,……
    \item[名 / \V\ftn{形} -i/ni teile] 仅……,只……:
    \begin{itemize}
        \item \irg bi - bisirei teile 倾其所有
        \item \bil{mutembi}{能够} - muterei teile 尽其所能  
    \end{itemize}
    \item[ai \V\ftn{形} babi.]有何可\V 之处?
    \item[名 -i/ni / \V\ftn{形} + gese / adali]像……一样
    \item[名 -i/ni / \V\ftn{形} + iqi] 沿着,顺着
\end{des}


\subsection{特殊例句辨析}

\paragraph{先后顺序的细微区别}

\begin{itemize}
    \item batu iliha manggi tuqike. 站起来之后出去了
    \item batu ili manggi tuqike. 站起来之后紧接着马上就出去了
    \item batu ilime saka / jaka tuqike. 站起来之后紧接着马上就出去了
    \item batu ili nakv tuheke. 站起来却摔倒了
\end{itemize}

\paragraph{定语从句} 定语从句的主语用属格:

bi (ini minde juwen buhe) bithe be inde be derebumbi.

其中,i minde \bil{juwen}{借} \bil{bu=he}{给-了} 做 bithe 的定语。

batu -i hvlaha bithe be, bi \bil{inu}{也} hvlaha bihe.

\pagebreak

\section{变格}

\begin{des}
    \item[\V =rakv ] 表否定。
    \item[\V =me] 并列副动词。
    \item[\V =ro / \V =ra / \V =re]  \lat{-rA}形,具体由第一和谐律选用。作用:
        \begin{enumerate}
            \item “随意”语气;
            \item 表示将要发生;
            \item 在现在将来时肯定句中与 \bil{uthai}{就} 搭配(对应地,与 \bil{urunakv}{一定} 搭配的句尾应为 =mbi 形);
            \item 作为形动词、动名词充当定语。
        \end{enumerate}
    \item[\V =ha / \V =he / \V =ho / \V =ka / \V =ke / \V =ko] \lat{-HA}形,表过去。由第二和谐律选用。
    \item[\V =nambi / \V =nembi / \V =nombi] 去做\V  ,依第二和谐律选用。
    \item[\V \bil{='jimbi}{\lat{-njimbi}}] 来做\V 。
    \item[\V =qambi / \V =qembi / \V =qombi] 一起做,依第二和谐律选用。
    \item[\V =ndumbi / \V =numbi] 互相做\V ,二者等价。
    \item[\V =qi] 如果做\V。
\end{des}

\subsection{\lat{-rA}形第一和谐律}

根据\V 结尾音节元音:

\begin{itemize}
    \item o 接 =ro;
    \item a 接 =ra;
    \item 其他接 =re。
\end{itemize}

总结为下表:

\begin{center}
    \begin{tabular}{c|c}
        \toprule
        末音节元音 & \lat{-rA形}\\
        \midrule
        a & \V =ra \\\hline
        e & \multirow{3}{*}{\V =re} \\\cline{1-1}
        i &  \\\hline
        o & \V =ro \\\hline
        u & \multirow{3}{*}{\V =re} \\\cline{1-1}
        v &  \\
        \bottomrule
    \end{tabular}
\end{center}
        
\subsection{\lat{-HA}形第二和谐律} 规则较为复杂。

\begin{enumerate}
    \item \V 为单音节且元音为a / o,接=ha。否则:
    \item \V 末音节元音为o,接=ho;
    \item \V 末音节元音为a,接=ha;
    \item \V 末音节元音为i / v,且倒数第二音节元音为a / i / o / u / v,接=ha;
    \item \V 末音节元音为u,且倒数第二音节元音为a / i / o / v,接=ha;
    \item \V 末音节为二合元音,且前一元音为a / i / o / v,接=ha。除非:
    \item \V 上附加表示使被动的 =bu= 时,仍沿用未添加 =bu= 时的后缀,仅有\V =ho 需变为\V =buha。
    \item 其余接=he。
    \item 特殊除外。
    % \begin{itemize}
    %     \item dosimbi - dosika
    %     \item wesimbi - wesike
    %     \item ombi - oho
    % \end{itemize}
\end{enumerate}

总结为下表:

\begin{center}
    \begin{tabular}{c|c|c|c|c|c|c|c|c}
    \toprule
    \multirow{4}{*}{末音节元音} & \multicolumn{8}{c}{\lat{-HA形}} \\
    % \mrow[3]{末音节元音} & \multicolumn{8}{c}\lat{-HA形} \\
    \cline{2-9} 
    & \multirow{3}{*}{作为双合元音的前者} &  \multicolumn{7}{c}{搭配前音节元音}                        \\ 
    \cline{3-9} 
                  & & 无  & a  & e & i  & o         & u   & v    \\\midrule
    a             &   \multicolumn{8}{c}{\V =ha}                            \\\hline
    e             &  \multicolumn{8}{c}{\V =he}      \\\hline
    i             &  \multirow{3}{*}{\V =ha}       &  \V =he  & \V =ha & \V =he  & \multicolumn{4}{c}{\V =ha}              \\\cline{1-1} \cline{3-9}
    o             &         & \V =ha & \multicolumn{6}{c}{\V =ho (使被动为\V =buha)}                   \\\hline
    u             &  \V =he       & \multirow{3}{*}{\V =he}  & \multirow{3}{*}{\V =ha} & \multirow{3}{*}{\V =he}  & \multicolumn{2}{c|}{\V =ha} &  \V =he   & \V =ha   \\\cline{1-2} \cline{6-9}
    v             &  \V =ha       &   &  &   & \multicolumn{4}{c}{\V =ha}   \\\bottomrule
    \end{tabular}
\end{center}

\subsection{特殊变格的单词总结}

\begin{itemize}
    \item \bil{jembi}{吃} - jeterakv
    \item \bil{ombi}{可以} - ojorakv
    \item \bil{bimbi}{有 / 存在} - bisirakv
    \item \bil{sambi}{知道} - sarkv
\end{itemize}

\subsection{特别小节:两大和谐律中的元音A选用表}

\begin{center}
    \begin{tabular}{c|c|c|c|c|c|c|c|c|c}
        \toprule
        \multirow{4}{*}{末音节元音} & \multirow{4}{*}{Ⅰ} & \multicolumn{8}{c}{Ⅱ} \\
        \cline{3-10} 
        & & \multirow{3}{*}{双合} & \multicolumn{7}{c}{搭配前音节元音} \\ 
        \cline{4-10} 
        & & & 无 & a & e & i & o & u & v \\\midrule
        a & \cellcolor{cyan} a & \cellcolor{cyan} a & \cellcolor{cyan} a & \cellcolor{cyan} a & \cellcolor{cyan} a & \cellcolor{cyan} a & \cellcolor{cyan} a & \cellcolor{cyan} a & \cellcolor{cyan} a \\\hline
        e & \cellcolor{lime} e & \cellcolor{lime} e & \cellcolor{lime} e & \cellcolor{lime} e & \cellcolor{lime} e & \cellcolor{lime} e & \cellcolor{lime} e & \cellcolor{lime} e & \cellcolor{lime} e \\\hline
        i & \cellcolor{lime} e & \cellcolor{cyan} a & \cellcolor{lime} e & \cellcolor{cyan} a & \cellcolor{lime} e & \cellcolor{cyan} a & \cellcolor{cyan} a & \cellcolor{cyan} a & \cellcolor{cyan} a \\\hline
        o* & \cellcolor{pink} o & \cellcolor{cyan} a & \cellcolor{cyan} a & \cellcolor{pink} o & \cellcolor{pink} o & \cellcolor{pink} o & \cellcolor{pink} o & \cellcolor{pink} o & \cellcolor{pink} o \\\hline
        u & \cellcolor{lime} e & \cellcolor{lime} e & \cellcolor{lime} e & \cellcolor{cyan} a & \cellcolor{lime} e & \cellcolor{cyan} a & \cellcolor{cyan} a & \cellcolor{lime} e & \cellcolor{cyan} a \\\hline
        v & \cellcolor{lime} e & \cellcolor{cyan} a & \cellcolor{lime} e & \cellcolor{cyan} a & \cellcolor{lime} e & \cellcolor{cyan} a & \cellcolor{cyan} a & \cellcolor{cyan} a & \cellcolor{cyan} a \\\bottomrule
    \end{tabular}
    
    *注意:使被动形式时需要格外关注o的变化。
\end{center}

% \begin{center}
%     \begin{tabular}{c|c|c|c|c|c|c|c|c|c}
%         \toprule
%         \multirow{4}{*}{末音节元音} & \multirow{4}{*}{第一和谐律} & \multicolumn{8}{c}{第二和谐律} \\
%         \cline{3-10} 
%         & & \multirow{3}{*}{作为双合元音的前者} & \multicolumn{7}{c}{搭配前音节元音} \\ 
%         \cline{4-10} 
%           & & & 无 & a & e & i & o & u & v \\\midrule
%         a & a & a & a & a & a & a & a & a & a \\\hline
%         e & e & e & e & e & e & e & e & e & e \\\hline
%         i & e & a & e & a & e & a & a & a & a \\\hline
%         o & o & a & a & o & o & o & o & o & o \\\hline
%         u & e & e & e & a & e & a & a & e & a \\\hline
%         v & e & a & e & a & e & a & a & a & a \\\bottomrule
%     \end{tabular}
% \end{center}


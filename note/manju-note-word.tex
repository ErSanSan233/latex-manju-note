\pagebreak

\section{词}

\subsection{助词}

\begin{des}
    \item[oqi] 提示助词。
    \item[de] 与位格格助词,在……、向……、对……。
    \item[be] 宾经格格助词,把……、将……、经过……,也可表目的。
    \begin{itemize}
        \item bi \bil{isan}{会议} de genembi. 
    \end{itemize}
    \item[-i/ni] 属用格格助词,……的、用……。
    \begin{itemize}
        \item aini整体认读:用什么。
    \end{itemize}
    \item[qi] 从比格格助词,从……。
    \item[deri] 从比格格助词,穿过。
\end{des}

\begin{itemize}
    \item batu aibi=\aux{=qi} ya ba \aux{be} aibi=\aux{=de} genembi?
    \item batu beging \aux{qi} simiyan \aux{be} halbin \aux{de} genembi.
    \item  bi dergi ba \aux{-i} \bil{amba}{大} tang gurun \aux{qi} wargi abka \aux{de} \bil{yargiyan}{真的} \bil{nomun}{经} \aux{be} \bil{bai=\!}{求}\bil{=me}{副动词} genembi.
\end{itemize}

\subsection{人称代词}

人称代词如下表:

\begin{center}
    \begin{tabular}{c|c|c}
        \toprule
        人称 & 单数 & 复数\\
        \midrule
        \multirow{3}{*}{第一} & \multirow{3}{*}{bi} & be(严格排除对方)\\
            &   &   muse(严格包含对方)\\
        \hline
        第二 & si & suwe\\
        \hline
        第三(正式、尊敬、仅指人) & i & qe\\
        \hline
        第三(随意、不敬、可指物) & tere & tese\\
        \bottomrule
    \end{tabular}
\end{center}

变格时,b= 会变成 n= ,且根据读音规律搭接鼻辅音,即 =be 前加 =m= 、=de / =i / =qi 前接 \bil{='=}{\lat{n}},双音节词除外。全表格如下,后续课文中有分写情况则单独纳入:

\begin{center}
    \begin{tabular}{c|c|c|c|c}
        \toprule
        \multirow{3}{*}{人称代词} & \multicolumn{4}{c}{格和格助词} \\
        \cline{2-5}
            & 属用 -i & 宾经 be & 与位 de & 从比 qi\\
        \midrule
        bi & mini & mimbe & minde & minqi \\
        be & meni & membe & mende & menqi \\
        \hline
        muse & musei & musebe / muse be & musede & museqi \\
        \hline
        si & sini & simbe & sinde & sinqi \\
        suwe & suweni & suwembe & suwende & suwenqi \\
        \hline
        i & ini & imbe & inde & inqi \\
        qe & qeni & qembe & qemde & qenqi \\
        \hline
        tere & terei & terebe & terede & tereqi \\
        tese & tesei & tesebe & tesede & teseqi \\
        \bottomrule
    \end{tabular}
\end{center}

\subsection{方位词}

\begin{center}
    \begin{tabular}{c|c|c|c}
        \toprule
       方位 & 不接 de 形式 & 接 de 形式 & =si 形式(向)\\
       \midrule
       内 & dolo & dorgi & dosi \\
       外 & tule & tulerge & tulesi \\
       上/东 & dele & dergi & desi \\
       下/西 & wala & wargi & wasi \\
       前/南 & juleri & julergi & julesi \\
       后/北 & amala & amargi & amasi \\
        \bottomrule
    \end{tabular}
\end{center}

\subsection{数词}
基数词(末尾去\lat{n})=ta / =te / =to变分配数词,其中juwan不去\lat{n}:juwanta

基数词接=nggeri ~次 / 回 / 遭:ilanggeri。\irg emu - emgeri

次数: 序数词 mudan : ilaqi mudan

\subsection{形容词}
形容词 + -i/ni 可变为副词,但一部分词不需要加,称作“兼类词”。

\paragraph{增强级}
ele:形容词比较级(增强级)。batu jai tondo, we \aux{ele} \bil{den}{高}? batu \aux{ele} den.

\paragraph{减弱级}
形 + \ii{=kan}{=ken}{=kon} 形容词减弱级:略……,稍微……:\\
\bil{lamun}{蓝} - lamukan / 
\bil{eshun}{生} - eshuken
形容词(去末尾=n)接=k=\AIImedi=n :略……,稍微 ……。
\begin{itemize}
    \item \bil{lamun}{蓝} - lamukan
    \item \bil{eshun}{生} - ashuken
\end{itemize}
\irg 特殊情况加 =si / =liyan / =meliyan / =shun / =shvn,而且可叠加:
\begin{itemize}
    \item \bil{ajige}{小} - ajigesi
    \item \bil{amba}{大} - ambakan - ambakasi
\end{itemize}

\subsection{特别小节:不规则动词整理}

\begin{des}
    \item[bimbi] bisirakv 
    \item[dosimbi] dosika 
    \item[jembi] jefu! - jeterakv - jekenembi / jekenjimbi - jeke / jebuhe
    \item[jimbi] jiu! - jiderakv
    \item[sambi] sarkv 
    \item[ombi] oso! - ojorakv - ojorou - oho
    \item[wesimbi] wesike 
    \item[fosombi]照射 - fosoko
    \item[tuqimbi] tuqike  
\end{des}

\subsection{特殊变格的单词总结}

\begin{itemize}
    \item \bil{jembi}{吃} - jeterakv
    \item \bil{ombi}{可以} - ojorakv
    \item \bil{bimbi}{有 / 存在} - bisirakv
    \item \bil{sambi}{知道} - sarkv
\end{itemize}
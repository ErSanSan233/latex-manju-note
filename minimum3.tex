\documentclass{article}
% \usepackage{xeCJK}
\usepackage{atbegshi}
\usepackage[normalem]{ulem}  % Provides underline and strikeout commands
\usepackage{tikz}            % Required for drawing the arrow
% \usepackage{multirow}
% \usepackage{booktabs}

\setmainfont[
    Mapping = abkai-to-manju,
    FakeStretch=1.1]
    {AbkaiXanyan}

\setCJKmainfont[
    RawFeature={vertical:+vert}, 
    Scale=0.7,
    FakeStretch=0.95]
    {NotoSerifCJKsc-Regular}

\newcommand*\CJKmovesymbol[1]{\raise.45em\hbox{#1}}
\newcommand*\CJKmove{\punctstyle{plain}
    \let\CJKsymbol\CJKmovesymbol
    \let\CJKpunctsymbol\CJKsymbol}
    \AtBeginDocument{\CJKmove}

\XeTeXupwardsmode1

\AtBeginShipout{%
    \global\setbox\AtBeginShipoutBox\vbox{%
        \special{pdf: put @thispage <</Rotate 90>>}%
        \box\AtBeginShipoutBox
    }%
}%

%箭头绘制
\newcommand{\arrowuline}[1]{%
    \tikz[baseline=(char.base)]{
        \node[inner sep=0pt,outer sep=0pt] (char) {#1};
        \draw[blue,latex-] ([yshift=0.5ex]char.north west) -- ([yshift=0.5ex]char.north east) |- (char.south east);
    }%
}

\newcommand{\aux}[1]{%
    \!\arrowuline{#1}\!%
}

%双语展示
\newcommand{\bil}[2]{%
    \raisebox{0.8em}[0pt][0pt]{\makebox[0cm][l]{\tiny #2}}\mbox{#1}%
}

%照抄拉丁字母
\setfontfamily\latin[
    Scale = 0.7]
    {NotoSerifCJKsc-Regular}

\newcommand{\lat}[1]{
        {\latin #1}
    }
    

\begin{document}
% batu beging \arrowuline{qi} simiyan \arrowuline{be} \\
% halbin \arrowuline{de} genembi.

% \noindent batu beging \arrowuline{qi} simiyan \arrowuline{be} \\
% halbin \arrowuline{de} genembi.

% \begin{itemize}
%     \item 封装前、无字干:aibi\arrowuline{qi}
%     \item 封装后、无字干:aibi\aux{qi}
%     \item 封装后、有字干:
% aibi=\aux{=qi}
% \end{itemize}

bi dergi ba \aux{-i} \bil{amba}{大} tang gurun \aux{qi} \\
wargi abka \aux{de} \bil{yargiyan}{真的} \bil{nomun}{经} \aux{be} \\
\bil{bai=\!}{求}\bil{=me}{副动词} genembi.

~\\

a / e / i / o / u 后接 \bil{=o}{\lat{-u}}

% en

% en'

% \begin{center}
%     \begin{tabular}{c|c|c}
%         \toprule
%         人称 & 单数 & 复数\\
%         \midrule
%         \multirow{3}{*}{第一} & \multirow{3}{*}{bi} & be(严格排除对方)\\
%             &   &   muse(严格包含对方)\\
%         \hline
%         第二 & si & suwe\\
%         \hline
%         第三(正式、尊敬、仅指人) & i & qe\\
%         \hline
%         第三(随意、不敬、可指物) & tere & tese\\
%         \bottomrule
%     \end{tabular}
% \end{center}
\end{document}